\documentclass{article}
\title{Posterior Regression: Approximated Statistically Approximate Distributions}
\author{Prasad N R\\Alumnus, Computer Engineering, \\Carnegie Mellon University\\prasadnr606@yahoo.in}

\begin{document}
	\maketitle 
	\begin{abstract}
		Explainable AI is an important aspect. But, to the best of knowledge, there isn't any end-to-end probabilistic explainable algorithms that is easy to work with. So, using Laplace smoothing idea of Bayes probabilities, an end-to-end probabilistic deep-learning algorithm is proposed. Also, there are three initial ideas related to the machine learning training process of this deep-learning.
	\end{abstract}
	
	\begin{section}{Introduction}
		Bayes probability rule provides us the posteriori. So, an idea is to use probabilities that are not dependent on specific probability distributions, Gaussian distribution for example. Also, usually Laplace smoothing is used and 1's are added to the Bayes conditional probabilities. But, this 1 may not be the best additive correction all the time. Also, a hierarchy of these probabilities is proposed. Regarding the probability training process, there seems to be none as this Approximated Bayes Approximate Distributions (ABAD) is an idea that is being proposed in this research-paper. 
	\end{section}
	
	\begin{section}{Prior Work}
		This work is inspired by Bayes statistics, Laplace smoothing and additive smoothing. Some of the convex optimisation solutions include Backpropagation and Blind-descent.\cite{hinton}\cite{blindDescent} Some of the ideas of explainable deep-learning is similar to the explanation algorithms of ABAD.\cite{explainableDeepLearning} Also, PyTorch is used for `autograd` of ABAD statistics.\cite{PyTorch} Although Scikit-learn has MNIST data-set and common functions, for ABAD, a custom Python forward-function has been designed.\cite{scikit-learn}
	\end{section}
	
	\begin{thebibliography}{1}
		
		\bibitem{blindDescent}
		Akshat Gupta and Prasad N R, \textit{“Blind Descent: A Prequel to Gradient Descent”}, Vol. 783 Lecture Notes in Electrical Engineering,
		ICDSMLA, Springer (2020)
		
		\bibitem{hinton}
		David E. Rumelhart, Geoffrey E. Hinton, Ronald J. Williams,
		\textit{Learning representations by back-propagating errors}, Vol. 323,
		Nature, 1986
		
		\bibitem{explainableDeepLearning}
		Gabrielle Ras, Ning Xie, Marcel Van Gerven and Derek Doran, \textit{"Explainable Deep Learning: A Field Guide for the Uninitiated"}, Journal of Artificial Intelligence Research [Vol. 73] (2022)
		
		\bibitem{PyTorch}
		Paszke, A., Gross, S., Massa, F., Lerer, A., Bradbury, J., Chanan, G., Killeen, T., Lin, Z., Gimelshein, N., Antiga, L., Desmaison, A., Kopf, A., Yang, E., DeVito, Z., Raison, M., Tejani, A., Chilamkurthy, S., Steiner, B., Fang, L., Bai, J., and Chintala, S. (2019). \textit{PyTorch: An Imperative Style, High-Performance Deep Learning Library [Conference paper].} Advances in Neural Information Processing Systems 32, 8024–8035. http://papers.neurips.cc/paper/9015-pytorch-an-imperative-style-high-performance-deep-learning-library.pdf
	
		\bibitem{scikit-learn}
		Pedregosa, F. and Varoquaux, G. and Gramfort, A. and Michel, V.
		and Thirion, B. and Grisel, O. and Blondel, M. and Prettenhofer, P.
		and Weiss, R. and Dubourg, V. and Vanderplas, J. and Passos, A. and
		Cournapeau, D. and Brucher, M. and Perrot, M. and Duchesnay, E., \textit{Scikit-learn: Machine Learning in Python},
		Journal of Machine Learning Research, volume 12, pages 2825--2830, year 2011
	
	\end{thebibliography}
\end{document}
